\documentclass[12pt, letterpaper]{article}
\usepackage[titletoc,title]{appendix}
\usepackage{color}
\usepackage{booktabs}
\usepackage[usenames,dvipsnames,svgnames,table]{xcolor}
\definecolor{dark-red}{rgb}{0.75,0.10,0.10}
\definecolor{bluish}{rgb}{0.05,0.05,0.85}

\usepackage[margin=1in]{geometry}
\usepackage[linkcolor=blue,
			colorlinks=true,
			urlcolor=blue,
			pdfstartview={XYZ null null 1.00},
			pdfpagemode=UseNone,
			citecolor={bluish},
			pdftitle={pareto_party}]{hyperref}

\usepackage[resetlabels,labeled]{multibib}
\newcites{SI}{SI References}
\usepackage{natbib}

\usepackage{float}

\usepackage{geometry} % see geometry.pdf on how to lay out the page. There's lots.
\geometry{letterpaper}               % This is 8.5x11 paper. Options are a4paper or a5paper or other... 
\usepackage{graphicx}                % Handles inclusion of major graphics formats and allows use of 
\usepackage{amsfonts,amssymb,amsbsy}
\usepackage{amsxtra}
\usepackage{verbatim}
\setcitestyle{round,semicolon,aysep={},yysep={;}}
\usepackage{setspace}		     % Permits line spacing control. Options are \doublespacing, \onehalfspace
\usepackage{sectsty}		     % Permits control of section header styles
\usepackage{pdflscape}
\usepackage{fancyhdr}		     % Permits header customization. See header section below.
\usepackage{url}                     % Correctly formats URLs with the \url{} tag
\usepackage{fullpage}		%1-inch margins
\usepackage{multirow}
\usepackage{rotating}
\setlength{\parindent}{3em}

\usepackage[T1]{fontenc}
\usepackage[bitstream-charter]{mathdesign}

\usepackage{chngcntr}
\usepackage{booktabs}
\usepackage{longtable}

\def\citeapos#1{\citeauthor{#1}'s (\citeyear{#1})}

\makeatother

\usepackage{footmisc}
\setlength{\footnotesep}{\baselineskip}
\makeatother
\renewcommand{\footnotelayout}{\normalsize \doublespacing}


% Caption
\usepackage[hang, font=small,skip=0pt, labelfont={bf}]{caption}
%\captionsetup[subtable]{font=small,skip=0pt}
\usepackage{subcaption}

% tt font issues
% \renewcommand*{\ttdefault}{qcr}
\renewcommand{\ttdefault}{pcr}

\setcounter{page}{0}

\usepackage{lscape}
\renewcommand{\textfraction}{0}
\renewcommand{\topfraction}{0.95}
\renewcommand{\bottomfraction}{0.95}
\renewcommand{\floatpagefraction}{0.40}
\setcounter{totalnumber}{5}
\makeatletter
\providecommand\phantomcaption{\caption@refstepcounter\@captype}
\makeatother

\title{Partisan Vision? Impact of Partisanship on Visual Evaluative Tasks}

\author{Carrie Roush\thanks{Carrie can be reached at, \href{mailto:carolyn.roush@gmail.com}{\texttt{carolyn.roush@gmail.com}}} \and, Gaurav Sood\thanks{Gaurav can be reached at, \href{mailto:gsood07@gmail.com}{\texttt{gsood07@gmail.com}}} \and Alex Theodoridis\thanks{Alex can be reached at \href{alexandertheodoridis@gmail.com}{\texttt{alexandertheodoridis@gmail.com}}}}

\begin{comment}

setwd(paste0(githubdir, "partisan_vision/ms/"))
tools::texi2dvi("partisan_vision.tex", pdf = TRUE, clean = TRUE)
setwd(githubdir)

\end{comment}

\begin{document}
\maketitle
\thispagestyle{empty}

\begin{abstract}

\noindent In the current era of partisan polarization, partisanship strongly colors partisans' worldview. But does it cause partisans to \textit{see} different things? We test the hypothesis using two different experiments and a survey. The data suggest that the effect is generally small.
\end{abstract}

\newpage

\doublespacing

Partisans are increasingly polarized \cite{IyengarSoodLelkes2012} with partisan cleavages outstripping some of the longer standing racial cleavages. In this paper, we explore whether polarization affects how partisans 'see.' We test the hypothesis with simple evaluative tasks. In particular, we field two survey experiments and a survey. We find that partisan bias is generally small.

\section{Data and Research Design}
To assess how partisans \textit{see}, we fielded two survey experiments on a nationally representative sample of people selected by YouGov \citep{rivers2007} as part of a Cooperative Congressional Election Study (CCES) module. In the first experiment, we presented people a short passage and asked them to count the number of errors in it. We manipulated the perceived party of the person writing the text (see Figure~\ref{fig:mistakes_dem} and Republicans \ref{fig:mistakes_rep}). In the second experiment, we showed people a photo of a parking lot and asked them to estimate the number of poorly parked cars. We manipulated which parties' members parked the car by manipulating the caption indicating where the photo was taken (see \ref{fig:Parking_Lot_Dems.png}).

We complimented the survey experiments with a partisan evaluative task on a survey. On a survey conducted on MTurk, we asked respondents to watch a short video and estimate how many people in the video were wearing masks. In particular, our directions were as follows: ``Please watch the following short (10-second) video. You will be asked a question about it on the next screen. ...How many people in the video were wearing masks?''

\section{Results}

As Table~\ref{tab:error_sum} shows, Democrats find 9.7 mistakes when they think the text is written by a Democrat compared to 9.9 mistakes when they think the text is written by a Republican. On the other hand, Republicans find 8.4 mistakes on average when they think the text is written by a Democrat and 8.1 mistakes when they think it is written by a Republican. And while the differences are consistent with partisan bias, the differences are small. The point is especially clear when you look at the medians, which are the same.

% latex table generated in R 4.1.2 by xtable 1.8-4 package
% Sun Nov 14 23:56:49 2021
\begin{table}[!htb]
\centering
\begin{tabular}{llrrr}
  \hline
pid3lean & Error\_split & avg & med & n \\ 
  \hline
Democrat     & DEM & 9.7 & 10.0 & 334 \\ 
  Democrat     & REP & 9.9 & 10.0 & 324 \\ 
  Independent  & DEM & 9.7 & 10.0 & 94 \\ 
  Independent  & REP & 9.2 & 9.0 & 110 \\ 
  Republican   & DEM & 8.4 & 8.0 & 253 \\ 
  Republican   & REP & 8.1 & 8.0 & 252 \\ 
   \hline
\end{tabular}
\caption{Average Number of Errors} 
\label{tab:error_sum}
\end{table}


We see slightly larger partisan bias in the experiment that manipulates which party's members parked the cars. Democrats on average believe there are 6.1 poorly parked cars outside the Democratic Party meeting while they think there are 8.7 poorly parked cars in front of the Republican Party meeting (see Table ~\ref{tab:parking_sum}). As above, the medians are much closer at 5 and 6 for Democratic Party meeting and Republican Party meeting respectively. Switching to Republicans, the gap is much narrower---Republicans on average think that there are 8.7 poorly parked cars in front of the Democratic Party meeting and 8.1 in front of the Republican Party meeting. The gap in medians is 1. The results from Independents make interpretation slightly complicated as Independents show a pronounced pro-Republican bias. One plausible explanation is `hidden Republicans' among Independents.

% latex table generated in R 4.1.2 by xtable 1.8-4 package
% Sun Nov 14 23:57:16 2021
\begin{table}[!htb]
\centering
\begin{tabular}{llrrr}
  \hline
pid3lean & UCMParking\_split & avg & med & n \\ 
  \hline
Democrat     & Democratic Party & 6.1 & 5.0 & 211 \\ 
  Democrat     & Republican Party & 8.7 & 6.0 & 212 \\ 
  Independent  & Democratic Party & 10.9 & 7.0 & 57 \\ 
  Independent  & Republican Party & 6.7 & 5.0 & 61 \\ 
  Republican   & Democratic Party & 8.4 & 6.0 & 181 \\ 
  Republican   & Republican Party & 8.1 & 5.0 & 163 \\ 
   \hline
\end{tabular}
\caption{Average Number of Errors} 
\label{tab:error_sum}
\end{table}


Turning to the results from the survey, we see that results are again muted. We would have expected large differences between Democrats and Republicans but instead we see that the medians are the same and at the 75th percentile, there is a difference of 1.

% latex table generated in R 4.2.2 by xtable 1.8-4 package
% Sun Nov 27 23:33:13 2022
\begin{table}[!htb]
\centering
\caption{Number of People Wearing Masks} 
\label{tab:trump_sum}
\begin{tabular}{lrrrrr}
  \hline
pid\_dem\_l & p\_25 & p\_50 & p\_75 & n & std\_error \\ 
  \hline
democrat & 1.0 & 1.0 & 5.0 & 919 & 3.0 \\ 
  independent & 1.0 & 1.0 & 5.0 & 111 & 3.0 \\ 
  republican & 2.0 & 2.0 & 6.0 & 879 & 4.0 \\ 
   \hline
\end{tabular}
\end{table}


\clearpage
\bibliographystyle{apsr}
\bibliography{vision}
\clearpage


\appendix
\renewcommand{\thesection}{SI \arabic{section}}
\setcounter{table}{0}\renewcommand\thetable{\thesection.\arabic{table}}  
\setcounter{figure}{0}\renewcommand\thefigure{\thesection.\arabic{figure}}
\counterwithin{figure}{section}

\section{Treatment Figures}
\begin{figure}[!htbp]
\centering
\caption{Share of Black Criminals in Law \& Order and the Real World}
\includegraphics[scale=.4]{../data/treats/Mistakes_Dem.png}
\label{fig:mistakes_dem}
\end{figure}

\begin{figure}[!htbp]
\centering
\caption{Mistakes}
\includegraphics[scale=.4]{../data/treats/Mistakes_Rep.png}
\label{fig:mistakes_rep}
\end{figure}

\begin{figure}[!htbp]
\centering
\caption{Parking Lot}
\includegraphics[scale=.4]{../data/treats/Parking_Lot_Dems.png}
\label{fig:mistakes_rep}
\end{figure}

\clearpage
\section{Figures}

\begin{figure}[!htbp]
\centering
\caption{Poorly Parked Cars}
\includegraphics[scale=.6]{../figs/error.pdf}
\label{fig:mistakes_rep}
\end{figure}


\begin{figure}[!htbp]
\centering
\caption{Poorly Parked Cars}
\includegraphics[scale=.6]{../figs/parking.pdf}
\label{fig:mistakes_rep}
\end{figure}


\end{document}